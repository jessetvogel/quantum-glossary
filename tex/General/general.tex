\begin{topic}{separable-state}{separable state}
    Let $\mathcal{H}_1, \ldots, \mathcal{H}_n$ be the Hilbert spaces of $n$ quantum systems. A state $\ket{\psi} \in \mathcal{H}_1 \otimes \cdots \otimes \mathcal{H}_n$ is \emph{separable} if it can be written as $\ket{\psi} = \ket{\psi_1} \otimes \cdots \otimes \ket{\psi_n}$ for some $\ket{\psi}_i \in \mathcal{H}_i$.
\end{topic}

\begin{topic}{entangled-state}{entangled state}
    Let $\mathcal{H}_1, \ldots, \mathcal{H}_n$ be the Hilbert spaces of $n$ quantum systems. A state $\ket{\psi} \in \mathcal{H}_1 \otimes \cdots \otimes \mathcal{H}_n$ is \emph{entangled} if it is not \tref{separable-state}{separable}, that is, if it cannot be written as $\ket{\psi} = \ket{\psi_1} \otimes \cdots \otimes \ket{\psi_n}$ for some $\ket{\psi}_i \in \mathcal{H}_i$.
\end{topic}

\begin{topic}{density-matrix}{density matrix}
    Let $\mathcal{H}$ be the Hilbert space of a quantum system. A \emph{density matrix}, also known as \emph{density operator}, on $\mathcal{H}$ is a bounded operator $\rho \colon \mathcal{H} \to \mathcal{H}$ which is positive-definite and has trace $\tr(\rho) = 1$.
\end{topic}

\begin{example}{density-matrix}
    In general, any density matrix $\rho$ can be written as
    \[ \rho = \sum_{i = 1}^{m} p_i \ket{\psi_i} \bra{\psi_i} \]
    for some $\ket{\psi_i} \in \mathcal{H}$ and $0 \le p_i \le 1$ with $\sum_{i = 1}^{m} p_i = 1$. If $p_i = 1$ for some $i$, then $\rho$ is called \tref{pure-state}{pure}. Otherwise, $\rho$ is called \tref{mixed-state}{mixed}.
\end{example}

\begin{topic}{pure-state}{pure state}
    Let $\mathcal{H}$ be the Hilbert space of a quantum system. A \emph{pure state} is a \tref{density-matrix}{density matrix} of the form $\rho = \ket{\psi} \bra{\psi}$ for some $\ket{\psi} \in \mathcal{H}$.
\end{topic}

\begin{topic}{mixed-state}{mixed state}
    Let $\mathcal{H}$ be the Hilbert space of a quantum system. A \emph{mixed state} is a \tref{density-matrix}{density matrix} which is not a \tref{pure-state}{pure state}.
\end{topic}

\begin{topic}{purity}{purity}
    The \emph{purity} of a \tref{density-matrix}{density matrix} $\rho$ is the trace of its square, that is, $\tr(\rho^2)$.
\end{topic}

\begin{example}{purity}
    When $\rho = \ket{\psi} \bra{\psi}$ is \tref{pure-state}{pure}, the purity of $\rho$ is $\tr(\rho^2) = \braket{\psi}{\psi}^2 = 1$. In general, the purity satisfies $(\dim \mathcal{H})^{-1} \le \tr(\rho^2) \le 1$.
\end{example}

\begin{example}{purity}
    The purity of a density matrix $\rho$ is conserved under unitary transformations. That is, for any unitary transformation $U$,
    \[ \tr((U \rho U^\dag)^2) = \tr(\rho^2) \]
    using that $UU^\dag = U^\dag U = 1$ and the cyclic property of the trace.
\end{example}

\begin{topic}{reduced-density-matrix}{reduced density matrix}
    Let $\mathcal{H}_1$ and $\mathcal{H}_2$ be the Hilbert spaces of two quantum systems. Given a \tref{density-matrix}{density matrix} $\rho$ on $\mathcal{H}_1 \otimes \mathcal{H}_2$, the \emph{reduced density matrix} of $\rho$ on $\mathcal{H}_1$ is the density matrix
    \[ \rho_1 = \tr_2(\rho) \]
    given by the partial trace over $\mathcal{H}_2$. Explicitly, the partial trace $\tr_2$ is given by
    \[ \tr_2 \left( \ket{\psi} \bra{\psi'} \otimes \ket{\varphi} \bra{\varphi'} \right) = \braket{\varphi'}{\varphi} \cdot \ket{\psi} \bra{\psi'} \]
    for all $\ket{\psi}, \ket{\psi'} \in \mathcal{H}_1$ and $\ket{\varphi}, \ket{\varphi'} \in \mathcal{H}_2$ and extended linearly.
\end{topic}

\begin{topic}{no-cloning-theorem}{no-cloning theorem}
    Consider two quantum systems, $A$ and $B$, with a common Hilbert space $\mathcal{H}_A = \mathcal{H}_B = \mathcal{H}$. The \emph{no-cloning theorem} states that there exists no unitary operation $U$ on $\mathcal{H}_A \otimes \mathcal{H}_B$ such that
    \[ U (\ket{\psi} \otimes \ket{\phi}) = \ket{\psi} \otimes \ket{\psi} \]
    for all $\ket{\psi} \in \mathcal{H}$ and any $\ket{\phi} \in \mathcal{H}$.
\end{topic}

\begin{example}{no-cloning-theorem}
    \begin{proof}
        If such an operation $U$ would exist, it would not be linear in $\ket{\psi}$, and hence would not be unitary.
    \end{proof}
\end{example}

\begin{topic}{uncertainty-principle}{uncertainty principle}
    Let $A$ and $B$ be two observables on a Hilbert space $\mathcal{H}$. For any state $\ket{\psi} \in \mathcal{H}$, denote by
    \[ \begin{aligned}
        \sigma_A^2 &= \bra{\psi} (A - \bra{\psi} A \ket{\psi})^2 \ket{\psi} = \bra{\psi} A^2 \ket{\psi} - \bra{\psi} A \ket{\psi}^2 \\
        \sigma_B^2 &= \bra{\psi} (B - \bra{\psi} B \ket{\psi})^2 \ket{\psi} = \bra{\psi} B^2 \ket{\psi} - \bra{\psi} B \ket{\psi}^2
    \end{aligned} \]
    the \textit{variance} of $A$ and $B$. The \emph{uncertainty principle} states that
    \[ \sigma_A^2 \sigma_B^2 \ge \left(\frac{1}{2 i} [A, B] \right)^2 \]
    for all states $\ket{\psi} \in \mathcal{H}$, where $[A, B] = AB - BA$ denotes the commutator of $A$ of $B$.
\end{topic}

\begin{example}{uncertainty-principle}
    For $A = x$ and $B = p = - i \hbar \tfrac{\partial}{\partial x}$ the position and momentum operators of a particle, the uncertainty principle reduces to the \textit{Heisenberg uncertainty principle}
    \[ \sigma_x^2 \sigma_p^2 \ge \left(\frac{1}{2 i} [x, p] \right)^2 = \frac{\hbar^2}{4} \]
    since $[x, p] = i \hbar$.
\end{example}

\begin{topic}{lenz-ising-model}{Lenz--Ising model}
    The \emph{Lenz--Ising model} is an \tref{COM:qubit}{$n$-qubit} quantum system with Hamiltonian
    \[ H = - \sum_{\substack{i, j = 1 \\ i \ne j}}^{n} J_{ij} \sigma_z^{(i)} \sigma_z^{(j)} - \mu \sum_{j = 1}^{n} h_j \sigma_z^{(j)} \]
    where
    \begin{itemize}
        \item $\sigma_z^{(i)}$ is $\left(\begin{smallmatrix} 1 & 0 \\ 0 & -1 \end{smallmatrix}\right)$ acting on the $i$-th qubit,
        \item $J_{ij}$ are the \textit{coupling strengths},
        \item $h_j$ are the \textit{external magnetic field strengths},
        \item $\mu$ is the \textit{magnetic moment}.
    \end{itemize}
\end{topic}
