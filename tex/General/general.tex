\begin{topic}{separable-state}{separable state}
    Let $\mathcal{H}_1, \ldots, \mathcal{H}_n$ be the Hilbert spaces of $n$ quantum systems. A state $\ket{\psi} \in \mathcal{H}_1 \otimes \cdots \otimes \mathcal{H}_n$ is \emph{separable} if it can be written as $\ket{\psi} = \ket{\psi_1} \otimes \cdots \otimes \ket{\psi_n}$ for some $\ket{\psi}_i \in \mathcal{H}_i$.
\end{topic}

\begin{topic}{entangled-state}{entangled state}
    Let $\mathcal{H}_1, \ldots, \mathcal{H}_n$ be the Hilbert spaces of $n$ quantum systems. A state $\ket{\psi} \in \mathcal{H}_1 \otimes \cdots \otimes \mathcal{H}_n$ is \emph{entangled} if it is not \tref{separable-state}{separable}, that is, if it cannot be written as $\ket{\psi} = \ket{\psi_1} \otimes \cdots \otimes \ket{\psi_n}$ for some $\ket{\psi}_i \in \mathcal{H}_i$.
\end{topic}

\begin{topic}{density-matrix}{density matrix}
    Let $\mathcal{H}$ be the Hilbert space of a quantum system. A \emph{density matrix}, also known as \emph{density operator}, on $\mathcal{H}$ is a bounded operator $\rho \colon \mathcal{H} \to \mathcal{H}$ which is positive-definite and has trace $\tr(\rho) = 1$.
\end{topic}

\begin{example}{density-matrix}
    In general, any density matrix $\rho$ can be written as
    \[ \rho = \sum_{i = 1}^{m} p_i \ket{\psi_i} \bra{\psi_i} \]
    for some $\ket{\psi_i} \in \mathcal{H}$ and $0 \le p_i \le 1$ with $\sum_{i = 1}^{m} p_i = 1$. If $p_i = 1$ for some $i$, then $\rho$ is called \tref{pure-state}{pure}. Otherwise, $\rho$ is called \tref{mixed-state}{mixed}.
\end{example}

\begin{topic}{pure-state}{pure state}
    Let $\mathcal{H}$ be the Hilbert space of a quantum system. A \emph{pure state} is a \tref{density-matrix}{density matrix} of the form $\rho = \ket{\psi} \bra{\psi}$ for some $\ket{\psi} \in \mathcal{H}$.
\end{topic}

\begin{topic}{mixed-state}{mixed state}
    Let $\mathcal{H}$ be the Hilbert space of a quantum system. A \emph{mixed state} is a \tref{density-matrix}{density matrix} which is not a \tref{pure-state}{pure state}.
\end{topic}

\begin{topic}{purity}{purity}
    The \emph{purity} of a \tref{density-matrix}{density matrix} $\rho$ is the trace of its square, that is, $\tr(\rho^2)$.
\end{topic}

\begin{example}{purity}
    When $\rho = \ket{\psi} \bra{\psi}$ is \tref{pure-state}{pure}, the purity of $\rho$ is $\tr(\rho^2) = \braket{\psi}{\psi}^2 = 1$. In general, the purity satisfies $(\dim \mathcal{H})^{-1} \le \tr(\rho^2) \le 1$.
\end{example}

\begin{example}{purity}
    The purity of a density matrix $\rho$ is conserved under unitary transformations. That is, for any unitary transformation $U$,
    \[ \tr((U \rho U^\dag)^2) = \tr(\rho^2) \]
    using that $UU^\dag = U^\dag U = 1$ and the cyclic property of the trace.
\end{example}

\begin{topic}{reduced-density-matrix}{reduced density matrix}
    Let $\mathcal{H}_1$ and $\mathcal{H}_2$ be the Hilbert spaces of two quantum systems. Given a \tref{density-matrix}{density matrix} $\rho$ on $\mathcal{H}_1 \otimes \mathcal{H}_2$, the \emph{reduced density matrix} of $\rho$ on $\mathcal{H}_1$ is the density matrix
    \[ \rho_1 = \tr_2(\rho) \]
    given by the partial trace over $\mathcal{H}_2$. Explicitly, the partial trace $\tr_2$ is given by
    \[ \tr_2 \left( \ket{\psi} \bra{\psi'} \otimes \ket{\varphi} \bra{\varphi'} \right) = \braket{\varphi'}{\varphi} \cdot \ket{\psi} \bra{\psi'} \]
    for all $\ket{\psi}, \ket{\psi'} \in \mathcal{H}_1$ and $\ket{\varphi}, \ket{\varphi'} \in \mathcal{H}_2$ and extended linearly.
\end{topic}

\begin{topic}{no-cloning-theorem}{no-cloning theorem}
    Consider two quantum systems, $A$ and $B$, with a common Hilbert space $\mathcal{H}_A = \mathcal{H}_B = \mathcal{H}$. The \emph{no-cloning theorem} states that there exists no unitary operation $U$ on $\mathcal{H}_A \otimes \mathcal{H}_B$ such that
    \[ U (\ket{\psi} \otimes \ket{\phi}) = \ket{\psi} \otimes \ket{\psi} \]
    for all $\ket{\psi} \in \mathcal{H}$ and any $\ket{\phi} \in \mathcal{H}$.
\end{topic}

\begin{example}{no-cloning-theorem}
    \begin{proof}
        If such an operation $U$ would exist, it would not be linear in $\ket{\psi}$, and hence would not be unitary.
    \end{proof}
\end{example}

\begin{topic}{qubit}{qubit}
    A \emph{qubit} is a quantum system with a two-dimensional Hilbert space $\mathcal{H}$. Its basis vectors are usually denoted by $\ket{0}$ and $\ket{1}$.
    
    An \emph{$n$-qubit quantum system} is a quantum system with Hilbert space $\mathcal{H}^{\otimes n}$. Its basis vectors are usually denoted by $\ket{b_1 b_2 \cdots b_n} = \ket{b_1} \otimes \ket{b_2} \otimes \cdots \otimes \ket{b_n}$.
\end{topic}

\begin{topic}{epr-pair}{Einstein--Podolsky--Rosen (EPR) pair}
    An \emph{Einstein--Podolsky--Rosen (EPR) pair} is a state in the \tref{qubit}{$2$-qubit} quantum system given by
    \[ \frac{1}{\sqrt{2}} (\ket{00} + \ket{11}) . \]
\end{topic}

\begin{example}{epr-pair}
    An EPR pair can be constructed using the following quantum circuit:
    \[ \svg \begin{quantikz}
        & \lstick{$|{0}\rangle$} & \gate{H} & \ctrl{1} & \qw
        \rstick[wires=2]{$\frac{1}{\sqrt{2}} \left( \ket{00} + \ket{11} \right)$} 
        \\
        & \lstick{$|{0}\rangle$} & \qw & \targ{} & \qw
    \end{quantikz} \]
    where $H$ denotes the \tref{ALG:hadamard-gate}{Hadamard gate} and $\oplus$ denotes the \tref{ALG:cnot-gate}{CNOT gate}.
\end{example}

\begin{topic}{ghz-state}{Greenberger--Horne--Zeilinger (GHZ) state}
    The \emph{Greenberger--Horne--Zeilinger (GHZ) state} is the \tref{qubit}{$3$-qubit} state given by
    \[ \frac{1}{\sqrt{2}} \left( \ket{000} + \ket{111} \right) . \]
    The \emph{generalized GHZ state} is the \tref{qubit}{$n$-qubit} state, for $n \ge 3$ given by
    \[ \frac{1}{\sqrt{2}} \left( \ket{0}^{\otimes n} + \ket{1}^{\otimes n} \right) . \]
\end{topic}

\begin{example}{ghz-state}
    The generalized GHZ state can be realized using a quantum circuit consisting of a \tref{ALG:hadamard-gate}{Hadamard gate} and a sequence of \tref{ALG:cnot-gate}{CNOT gates}. For instance, the following quantum circuit realizes the GHZ state for $n = 4$:
    \[ \svg \begin{quantikz}
        \lstick{$\ket{0}$} & \gate{H} & \ctrl{1} & \qw & \qw & \qw \\
        \lstick{$\ket{0}$} & \qw & \targ{} & \ctrl{1} & \qw & \qw \\
        \lstick{$\ket{0}$} & \qw & \qw & \targ{} & \ctrl{1} & \qw \\
        \lstick{$\ket{0}$} & \qw & \qw & \qw & \targ{} & \qw
    \end{quantikz} \]
\end{example}
