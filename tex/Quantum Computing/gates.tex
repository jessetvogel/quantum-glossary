\begin{topic}{hadamard-gate}{Hadamard gate}
    The \emph{Hadamard gate} is the \tref{qubit}{$1$-qubit} gate $H$ given by
    \[ \begin{aligned}
        H \ket{0} &= \frac{1}{\sqrt{2}} (\ket{0} + \ket{1}) \\
        H \ket{1} &= \frac{1}{\sqrt{2}} (\ket{0} - \ket{1}) .
    \end{aligned} \]
    In quantum circuits, the Hadamard gate is usually depicted as follows:
    \[ \svg \begin{quantikz}
        \qw & \gate{H} & \qw
    \end{quantikz} \]
\end{topic}

\begin{topic}{cnot-gate}{CNOT gate}
    The \emph{CNOT gate}, also known as the \emph{controlled NOT gate}, is the \tref{qubit}{$2$-qubit} gate given by
    \[ \textup{CNOT} (\ket{b_1} \otimes \ket{b_2}) = \ket{b_1} \otimes \ket{b_1 \oplus b_2} \]
    for $b_1, b_2 \in \{ 0, 1 \}$. In quantum circuits, the CNOT gate is usually depicted as follows:
    \[ \svg \begin{quantikz}
        \qw & \ctrl{1} & \qw \\
        \qw & \targ{} & \qw
    \end{quantikz} \]
\end{topic}

\begin{topic}{toffoli-gate}{Toffoli gate}
    The \emph{Toffoli gate}, also known as the \emph{CCNOT gate}, is the \tref{qubit}{$3$-qubit} gate given by
    \[ \begin{aligned}
        \textup{CCNOT} (\ket{00} \otimes \ket{b}) &= \ket{00} \otimes \ket{b} \\
        \textup{CCNOT} (\ket{01} \otimes \ket{b}) &= \ket{01} \otimes \ket{b} \\
        \textup{CCNOT} (\ket{10} \otimes \ket{b}) &= \ket{10} \otimes \ket{b} \\
        \textup{CCNOT} (\ket{11} \otimes \ket{b}) &= \ket{11} \otimes \ket{1 - b}
    \end{aligned} \]
    for $b \in \{ 0, 1 \}$. In quantum circuits, the Toffoli gate is usually depicted as follows:
    \[ \svg \begin{quantikz}
        \qw & \ctrl{2} & \qw \\
        \qw & \ctrl{1} & \qw \\
        \qw & \targ{} & \qw
    \end{quantikz} \]
\end{topic}

\begin{topic}{quantum-fourier-transform}{Quantum Fourier transform}
    For $N = 2^n$, the \emph{$N$-quantum Fourier transform} is the \tref{qubit}{$n$-qubit} gate given by the unitary operation
    \[ F_N = \frac{1}{\sqrt{N}} \begin{pmatrix}
        \omega_N^{0, 0} & \cdots & \omega_N^{0, N - 1} \\
        \vdots & \ddots & \vdots \\
        \omega_N^{N - 1, 0} & \cdots & \omega_N^{N - 1, N - 1}
    \end{pmatrix} \]
    with $\omega_N^{j, k} = \exp(2 \pi i j k / N)$ for all $0 \le j, k \le N - 1$.
\end{topic}

\begin{example}{quantum-fourier-transform}
    For $N = 2$, the quantum Fourier transform reduces to the \tref{hadamard-gate}{Hadamard gate}
    \[ F_N = \begin{pmatrix} 1 & 1 \\ 1 & -1 \end{pmatrix} = H . \]
\end{example}

\begin{topic}{fredkin-gate}{Fredkin gate}
    The \emph{Fredkin gate}, also known as the \emph{controlled swap gate}, is a \tref{qubit}{$3$-qubit} gate given by
    \[ \begin{aligned}
        F \ket{0} \otimes \ket{b_1 b_2} &= \ket{0} \otimes \ket{b_1 b_2} \\
        F \ket{1} \otimes \ket{b_1 b_2} &= \ket{1} \otimes \ket{b_2 b_1}
    \end{aligned} \]
    for $b_1, b_2 \in \{ 0, 1 \}$. In quantum circuits, the Fredkin gate is usually depicted as follows:
    \[ \svg \begin{quantikz}
        \qw & \ctrl{2} & \qw \\
        \qw & \swap{1} & \qw \\
        \qw & \targX{} & \qw
    \end{quantikz} \]
\end{topic}

\begin{topic}{pauli-gates}{Pauli gates}
    The \emph{Pauli $X$-, $Y$- and $Z$-gate} are three \tref{qubit}{$1$-qubit} gates that act via the Pauli matrices, that is,
    \[ X = \begin{pmatrix} 0 & 1 \\ 1 & 0 \end{pmatrix}, \quad Y = \begin{pmatrix} 0 & -i \\ i & 0 \end{pmatrix}, \quad Z = \begin{pmatrix} 1 & 0 \\ 0 & -1 \end{pmatrix} \]
    with respect to the standard basis given by $\ket{0}$ and $\ket{1}$. In quantum circuits, the Pauli gates are usually depicted as follows:
    \[ \svg \begin{quantikz} \qw & \gate{X} & \qw \end{quantikz} \begin{quantikz} \qw & \gate{Y} & \qw \end{quantikz} \begin{quantikz} \qw & \gate{Z} & \qw \end{quantikz} \]
\end{topic}

\begin{topic}{phase-shift-gates}{phase shift gates}
    A \emph{phase shift gate} is a \tref{qubit}{$1$-qubit} gate of the form
    \[ P(\phi) = \begin{pmatrix} 1 & 0 \\ 0 & e^{i \phi} \end{pmatrix} \]
    for some $\phi \in [0, 2 \pi]$, with respect to the standard basis given by $\ket{0}$ and $\ket{1}$.

    In particular, $S = P(\pi/2)$ is known as the \emph{$S$-gate} and $T = P(\pi/4)$ is known as the \emph{$T$-gate}.
\end{topic}

\begin{topic}{rotation-operator-gates}{Rotation operator gates}
    The \emph{rotation operator gates} are the three families of \tref{qubit}{$1$-qubit} gates given by
    \[ \begin{aligned}
        R_X(\theta) &= \exp(-i X \theta / 2) =  \begin{pmatrix} \cos(\theta) / 2 & - i \sin(\theta / 2) \\ - i \sin(\theta / 2) & \cos(\theta) \end{pmatrix} \\[15pt]
        R_Y(\theta) &= \exp(-i Y \theta / 2) =  \begin{pmatrix} \cos(\theta) / 2 & - \sin(\theta / 2) \\ \sin(\theta / 2) & \cos(\theta) \end{pmatrix} \\[15pt]
        R_Z(\theta) &= \exp(-i Z \theta / 2) =  \begin{pmatrix} \exp(-i \theta / 2) & 0 \\ 0 & \exp(i \theta / 2) \end{pmatrix}
    \end{aligned} \]
    where $\theta \in [0, 4\pi]$, with respect to the standard basis given by $\ket{0}$ and $\ket{1}$.
\end{topic}

\begin{topic}{clifford-gate}{Clifford gate}
    A \emph{Clifford gate} is any \tref{qubit}{$n$-qubit} gate $C$ such that
    \[ C P_n C^\dagger = P_n \]
    where $P_n$ is the \tref{pauli-group}{Pauli group} on $n$-qubits.
\end{topic}

\begin{topic}{pauli-group}{Pauli group}
    The \emph{Pauli group} on $n$ \tref{qubit}{qubits} is the \href{/math-definitions/#GT:group}{group} $P_n$ generated by $\pm 1, \pm i$ and all $n$-qubit gates of the form
    \[ \sigma_1 \otimes \sigma_2 \otimes \cdots \otimes \sigma_n \]
    for $\sigma_1, \ldots, \sigma_n \in \{ I, X, Y, Z \}$ where $X, Y, Z$ are the \tref{pauli-gates}{Pauli gates}.
\end{topic}
