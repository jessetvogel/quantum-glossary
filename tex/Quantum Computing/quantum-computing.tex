\begin{topic}{qubit}{qubit}
    A \emph{qubit} is a quantum system with a two-dimensional Hilbert space $\mathcal{H}$. Its basis vectors are usually denoted by $\ket{0}$ and $\ket{1}$.
    
    An \emph{$n$-qubit quantum system} is a quantum system with Hilbert space $\mathcal{H}^{\otimes n}$. Its basis vectors are usually denoted by $\ket{b_1 b_2 \cdots b_n} = \ket{b_1} \otimes \ket{b_2} \otimes \cdots \otimes \ket{b_n}$.
\end{topic}

\begin{topic}{epr-pair}{Einstein--Podolsky--Rosen (EPR) pair}
    An \emph{Einstein--Podolsky--Rosen (EPR) pair} is a state in the \tref{qubit}{$2$-qubit} quantum system given by
    \[ \frac{1}{\sqrt{2}} (\ket{00} + \ket{11}) . \]
\end{topic}

\begin{example}{epr-pair}
    An EPR pair can be constructed using the following quantum circuit:
    \[ \svg \begin{quantikz}
        \lstick{$\ket{0}$} & \gate{H} & \ctrl{1} & \qw
        \rstick[wires=2]{$\frac{1}{\sqrt{2}} \left( \ket{00} + \ket{11} \right)$} 
        \\
        \lstick{$\ket{0}$} & \qw & \targ{} & \qw
    \end{quantikz} \]
    where $H$ denotes the \tref{hadamard-gate}{Hadamard gate} and $\oplus$ denotes the \tref{cnot-gate}{CNOT gate}.
\end{example}

\begin{topic}{ghz-state}{Greenberger--Horne--Zeilinger (GHZ) state}
    The \emph{Greenberger--Horne--Zeilinger (GHZ) state} is the \tref{qubit}{$3$-qubit} state given by
    \[ \frac{1}{\sqrt{2}} \left( \ket{000} + \ket{111} \right) . \]
    The \emph{generalized GHZ state} is the \tref{qubit}{$n$-qubit} state, for $n \ge 3$ given by
    \[ \frac{1}{\sqrt{2}} \left( \ket{0}^{\otimes n} + \ket{1}^{\otimes n} \right) . \]
\end{topic}

\begin{example}{ghz-state}
    The generalized GHZ state can be realized using a quantum circuit consisting of a \tref{hadamard-gate}{Hadamard gate} and a sequence of \tref{cnot-gate}{CNOT gates}. For instance, the following quantum circuit realizes the GHZ state for $n = 4$:
    \[ \svg \begin{quantikz}
        \lstick{$\ket{0}$} & \gate{H} & \ctrl{1} & \qw & \qw & \qw \\
        \lstick{$\ket{0}$} & \qw & \targ{} & \ctrl{1} & \qw & \qw \\
        \lstick{$\ket{0}$} & \qw & \qw & \targ{} & \ctrl{1} & \qw \\
        \lstick{$\ket{0}$} & \qw & \qw & \qw & \targ{} & \qw
    \end{quantikz} \]
\end{example}

\begin{topic}{bell-states}{Bell's states}
    The \emph{Bell's states} are the four \tref{qubit}{$2$-qubit} states
    \[ \begin{aligned}
        \ket{\Phi^+} &= \frac{1}{\sqrt{2}} (\ket{00} + \ket{11}) \\
        \ket{\Phi^-} &= \frac{1}{\sqrt{2}} (\ket{00} - \ket{11}) \\
        \ket{\Psi^+} &= \frac{1}{\sqrt{2}} (\ket{01} + \ket{10}) \\
        \ket{\Psi^-} &= \frac{1}{\sqrt{2}} (\ket{01} - \ket{10}) .
    \end{aligned} \]
    They form a basis for the $2$-qubit system, called the \textit{Bell basis}.
\end{topic}

\begin{topic}{ancilla-qubit}{ancilla qubit}
    In a quantum circuit, an \emph{ancilla qubit} is an auxiliary \tref{qubit}{qubit}, that is, a qubit which is not used as input or output, but only for intermediate computations.
\end{topic}

% \begin{example}{ancilla-qubit}
%     Ancilla qubits can be used to avoid measurments.
% \end{example}

% \begin{example}{ancilla-qubit}
%     Ancilla qubits are sometimes necessary to keep the circuit reversible.
% \end{example}

\begin{topic}{quantum-signal-processing}{Quantum Signal Processing (QSP)}
    For any $a = \cos \theta \in [-1, 1]$, let $W(a) = \exp(i \theta X) = \begin{pmatrix} a & i \sqrt{1 - a^2} \\ i \sqrt{1 - a^2} & a \end{pmatrix}$ be a unitary gate acting on one \tref{qubit}{qubit}.
    A \emph{Quantum Signal Processing (QSP) sequence} of length $n \ge 0$ is a quantum circuit of the form
    \[ U(a, \phi_0, \ldots, \phi_n) = \exp(i \phi_0 Z) W(a) \exp(i \phi_1 Z) W(a) \cdots W(a) \exp(i \phi_n Z) \]
    for some $\phi_0, \ldots, \phi_n \in \RR$.
\end{topic}

\begin{example}{quantum-signal-processing}
    \textbf{Theorem} (\href{https://arxiv.org/pdf/2105.02859}{Martyn et al.})
    For any $\phi_0, \ldots, \phi_n \in \RR$, one has
    \[ U(a, \phi_0, \ldots, \phi_n) = \begin{pmatrix}
        P(a) & i Q(a) \sqrt{1 - a^2} \\
        i Q(a)^* \sqrt{1 - a^2} & P(a)^*
    \end{pmatrix} \quad \textup{ for all } a \in [-1, 1] \]
    for some polynomials $P$ and $Q$. Conversely, such $\phi_0, \ldots, \phi_n \in \RR$ exist for any polynomials $P$ and $Q$ satisfying
    \begin{itemize}
        \item $\deg P \le n$ and $\deg Q \le n - 1$,
        \item $P(-a) = (-1)^n P(a)$ and $Q(-a) = (-1)^{n - 1} Q(a)$,
        \item $|P|^2 + (1 - a)^2 |Q|^2 = 1$.
    \end{itemize}
\end{example}

\begin{example}{quantum-signal-processing}
    \begin{itemize}
        \item For $\phi_0 = \phi_1 = \phi_2 = 0$, we have $U(a, 0, 0, 0) = \begin{pmatrix}
            2 a^2 - 1 & 2 i a \sqrt{1 - a^2} \\
            2 i a \sqrt{1 - a^2} & 2 a^2 - 1
        \end{pmatrix}$.
        \item For $\phi_0 = \phi_1 = \phi_2 = \phi_3 = 0$, we have $U(a, 0, 0, 0, 0) = \begin{pmatrix}
            4 a^3 - 3 a & i (4 a^2 - 1) \sqrt{1 - a^2} \\
            i (4 a^2 - 1) \sqrt{1 - a^2} & 4 a^2 - 3 a
        \end{pmatrix}$.
        \item Let $(\phi_0, \phi_1, \phi_2, \phi_3, \phi_4, \phi_5) = (\pi/2, -\eta, 2 \eta, 0, -2 \eta, \eta)$ where $\eta = \tfrac{1}{2} \cos^{-1}(-1/4)$. Then
        \[ U(a, \phi_0, \ldots, \phi_5) = \begin{pmatrix}
            16 a^5 - 20 a^3 + 5 a & i (16 a^4 - 12 a^2 + 1) \sqrt{1 - a^2} \\
            i (16 a^4 - 12 a^2 + 1) \sqrt{1 - a^2} & 16 a^5 - 20 a^3 + 5 a
        \end{pmatrix} \]
        up to a global phase factor. This sequence is known as the \textit{BB1 pulse sequence}.
    \end{itemize}
\end{example}

\begin{topic}{quantum-singular-value-transformation}{Quantum Singular Value Transformation (QSVT)}
    Let $A$ be a real or complex matrix, and let $U$ be a block-encoding of $A$, that is, a unitary matrix of the form $U = \begin{pmatrix} A & * \\ * & * \end{pmatrix}$. Write $\Pi$ and $\tilde{\Pi}$ for the projections such that $A = \tilde{\Pi} U \Pi$.
    A \emph{Quantum Singular Value Transformation (QVST) sequence} of length $n \ge 0$ is a quantum circuit of the form
    \[ \tilde{\Pi}_{\phi_0} U \left[ \prod_{k = 1}^{(n - 1) / 2} \Pi_{\phi_{2k - 1}} U^\dagger \tilde{\Pi}_{\phi_{2k}} U \right] \Pi_{\phi_n} \quad \textup{ if $n$ is odd}, \]
    or
    \[ \left[ \prod_{k = 1}^{n / 2} \Pi_{\phi_{2k - 2}} U^\dagger \tilde{\Pi}_{\phi_{2k - 1}} U \right] \Pi_{\phi_n} \quad \textup{ if $n$ is even} \]
    for some $\phi_0, \ldots, \phi_n \in \RR$, where $\Pi_\phi = \exp(i \phi (2 \Pi - I))$ and $\tilde{\Pi}_\phi = \exp(i \phi (2 \tilde{\Pi} - 1))$.

    Writing $A = W \Sigma V^\dagger$ for the singular value decomposition of $A$, it can be shown that the QVST sequence takes the form
    \[ \begin{pmatrix} P(A) & * \\ * & * \end{pmatrix} \]
    for some polynomial $P$ of degree $n$, where $P(A) = W P(\Sigma) V^\dagger$ (that is, $P$ is applied to the singular values).
\end{topic}

\begin{example}{quantum-singular-value-transformation}
    When $A$ is a scalar, QSVT reduces to \tref{quantum-signal-processing}{QSP}.
\end{example}

\begin{topic}{quantum-tomography}{quantum tomography}
    \emph{Quantum tomography} is the process of reconstructing a quantum state using measurements on an ensemble of identical quantum states.
\end{topic}

\begin{example}{quantum-tomography}
    Let $\rho$ be a \tref{qubit}{$1$-qubit} \tref{GEN:mixed-state}{mixed state}. Since the \tref{pauli-gates}{Pauli matrices} $X, Y$ and $Z$, together with the identity matrix $I$, form an orthonormal basis for the space of $2 \times 2$ matrices with respect to the inner product $(A, B) \mapsto \tfrac{1}{2} \tr(A B)$, we can write
    \[ \rho = \frac{1}{2} \Big( \tr(\rho) I + \tr(\rho X) X + \tr(\rho Y) Y + \tr (\rho Z) Z \Big) . \]
    By measuring the observable $A \in \{ I, X, Y, Z \}$ for this state repeatedly, obtaining outcomes $a_1, \ldots, a_m$, one can estimate $\tr(\rho A)$ as the average $\overline{a} = \frac{1}{m} \sum_{i = 1}^{m} a_i$. This estimate has a standard deviation of at most $1 / \sqrt{m}$.

    This \textit{Pauli basis measurement tomography algorithm} generalizes to arbitrary $n$-qubit mixed states $\rho$, which can be expressed as
    \[ \rho = \frac{1}{2^n} \sum_{\sigma \in \{ I, X, Y, Z \}^n} \tr \big( (\sigma_1 \otimes \cdots \otimes \sigma_n) \rho \big) \cdot (\sigma_1 \otimes \cdots \otimes \sigma_n) \rho . \]
    By measuring the observables $\sigma_1 \otimes \cdots \otimes \sigma_n$, for $\sigma \in \{ I, X, Y, Z \}^n$, repeatedly, the state $\rho$ can be reconstructed.
\end{example}

\begin{topic}{phase-kickback}{phase kickback}
    \emph{Phase kickback} refers to the notion that the effect of \tref{controlled-gate}{controlled gates} can be seen as an effect on their controls, rather than an effect on their targets, and that these effects correspond to phasing operations.
\end{topic}

\begin{example}{phase-kickback}
    Consider the following circuit:
    \[ \svg \begin{quantikz}
        \lstick{$\ket{0}$} & \gate{H} & \ctrl{1} & \qw \\
        \lstick{$\ket{0}$} & \gate{X} & \gate{P(\phi)} & \qw
    \end{quantikz} \]
    After applying the \tref{hadamard-gate}{Hadamard gate} and \tref{pauli-gates}{$X$-gate}, we obtain the state $\tfrac{1}{\sqrt{2}} \left( \ket{0} + \ket{1} \right) \otimes \ket{1}$. After applying the controlled \tref{phase-shift-gates}{$P(\phi)$ gate}, we obtain the state
    \[ \frac{1}{\sqrt{2}} \left( \ket{01} + e^{i \phi} \ket{11} \right) = \frac{1}{\sqrt{2}} \left( \ket{0} + e^{i \phi} \ket{1} \right) \otimes \ket{1} \]
    Note that, even though the phase shift gate was applied to the second qubit, the effect is equivalent to as if it was applied to the first qubit. That is, the phase `kicked back' to the first qubit.
\end{example}

% \begin{example}{phase-kickback}
%     Consider the effect of a \tref{cnot-gate}{CNOT gate} where the target qubit is in the state $\tfrac{1}{\sqrt{2}} \left( \ket{0} - \ket{1} \right)$:
%     \[ \textup{CNOT} \left( \ket{0} \otimes \tfrac{1}{\sqrt{2}} \left( \ket{0} - \ket{1} \right) \right) = + \ket{0} \otimes \tfrac{1}{\sqrt{2}} \left( \ket{0} - \ket{1} \right) \]
%     \[ \textup{CNOT} \left( \ket{1} \otimes \tfrac{1}{\sqrt{2}} \left( \ket{0} - \ket{1} \right) \right) = - \ket{1} \otimes \tfrac{1}{\sqrt{2}} \left( \ket{0} - \ket{1} \right) \]
%     In particular, the effect of the CNOT gate is equivalent to the effect of a \tref{pauli-gates}{$Z$-gate} on the control qubit.
% \end{example}

\begin{example}{phase-kickback}
    Let $U$ be a unitary gate whose eigenvalues are $\pm 1$. Measurement of $U$ may be achieved as follows, using a controlled $U$-gate and phase kickback.
    \[ \svg \begin{quantikz}
        \lstick{$\ket{0}$} & \gate{H} & \ctrl{1} & \gate{H} & \meter{} \\
        \lstick[3]{$\ket{\Psi}$} & \qw & \gate[3]{U} & \qw & \qw \\
        & \qw & & \qw & \qw \\
        & \qw & & \qw & \qw
    \end{quantikz} \]
    To see why this works, write the initial state $\ket{\Psi}$ as $\alpha \ket{\psi_+} + \beta \ket{\psi_-}$ where $\ket{\psi_\pm}$ are eigenstates of $U$ with eigenvalue $\pm 1$, respectively. Now, as in the above circuit, apply a Hadamard gate to $\ket{0} \otimes \ket{\Psi}$ to obtain the state
    \[ \frac{1}{\sqrt{2}} \left( \ket{0} + \ket{1} \right) \otimes \left( \alpha \ket{\psi_+} + \beta \ket{\psi_-} \right)\]
    followed by a controlled $U$-gate to obtain \[ \frac{1}{\sqrt{2}} \left( \ket{0} \otimes \left( \alpha \ket{\psi_+} + \beta \ket{\psi_-} \right) + \ket{1} \otimes \left( \alpha \ket{\psi_+} - \beta \ket{\psi_-} \right) \right) \]
    and finally another Hadamard gate to obtain
    \[ \alpha \ket{0} \otimes \ket{\psi_+} + \beta \ket{1} \otimes \ket{\psi_-} . \]
    Measuring the first qubit will result in $+1$ with probability $|\alpha|^2$ and in $-1$ with probability $|\beta|^2$. 
\end{example}

\begin{topic}{shor-code}{Shor code}
    The \emph{Shor code} is a quantum error correcting code that encodes the logical \tref{qubit}{qubit} states $\ket{0}_L$ and $\ket{1}_L$ as the $9$-qubit codeword states
    \[ \begin{aligned}
        \ket{0}_L &= \frac{1}{\sqrt{8}} \left( \ket{000} + \ket{111} \right) \otimes \left( \ket{000} + \ket{111} \right) \otimes \left( \ket{000} + \ket{111} \right) , \\
        \ket{1}_L &= \frac{1}{\sqrt{8}} \left( \ket{000} - \ket{111} \right) \otimes \left( \ket{000} - \ket{111} \right) \otimes \left( \ket{000} - \ket{111} \right) .
    \end{aligned} \]
    The Shor code can detect and correct any $1$-qubit error given by a unitary of the form
    \[ E = \sum_{P \in \mathcal{P}} \alpha_P \cdot P \]
    where $\alpha_P \in \CC$ and $\mathcal{P} = \{ I, X^{(1)}, Y^{(1)}, Z^{(1)}, \ldots, X^{(9)}, Y^{(9)}, Z^{(9)} \}$ consists of the $1$-qubit \tref{pauli-gates}{Pauli gates} and identity.
    
    To detect and correct an error $E$ applied to an arbitrary logical state $\ket{\psi}_L = \alpha \ket{0}_L + \beta \ket{1}_L$, measure bits $s_1, \ldots, s_8$ using the circuit
    \[ \svg \begin{quantikz}
        \lstick{$E \ket{\psi}_L$} & \qw & \gate{S_i} & \qw \\
        \lstick{$\ket{0}$} & \gate{H} & \ctrl{-1} & \gate{H} & \metercw{s_i}
    \end{quantikz} \]
    where
    \[ \begin{gathered}
        S_1 = Z^{(1)} Z^{(2)}, \qquad S_2 = Z^{(2)} Z^{(3)}, \qquad S_3 = Z^{(4)} Z^{(5)}, \\[5pt]
        S_4 = Z^{(5)} Z^{(6)}, \qquad S_5 = Z^{(7)} Z^{(8)}, \qquad S_6 = Z^{(8)} Z^{(9)}, \\[5pt]
        S_7 = X^{(1)} X^{(2)} X^{(3)} X^{(4)} X^{(5)} X^{(6)}, \\[5pt]
        S_8 = X^{(4)} X^{(5)} X^{(6)} X^{(7)} X^{(8)} X^{(9)} .
    \end{gathered} \]
    One can show that the resulting state (before measurement of the $s_i$) is given by
    \[ \sum_{P \in \mathcal{P}} \alpha_P \cdot P \ket{\psi}_L \otimes \ket{s_{P, 1} \cdots s_{P, 8}} \]
    where the bitstring $s_P = s_{P, 1} \cdots s_{P, 8}$, called the \textit{syndrome} of $P$, is given by the following table.
    \[ \def\arraystretch{1.25} \begin{array}{cc|cc|cc}
         \hline P & s_P & P & s_P & P & s_P \\ \hline
         X^{(1)} & 10000000 & Y^{(1)} & 10000010 & Z^{(1)} & 00000010 \\
         X^{(2)} & 11000000 & Y^{(2)} & 11000010 & Z^{(2)} & 00000010 \\
         X^{(3)} & 01000000 & Y^{(3)} & 01000010 & Z^{(3)} & 00000010 \\
         X^{(4)} & 00100000 & Y^{(4)} & 00100011 & Z^{(4)} & 00000011 \\
         X^{(5)} & 00110000 & Y^{(5)} & 00110011 & Z^{(5)} & 00000011 \\
         X^{(6)} & 00010000 & Y^{(6)} & 00010011 & Z^{(6)} & 00000011 \\
         X^{(7)} & 00001000 & Y^{(7)} & 00001001 & Z^{(7)} & 00000001 \\
         X^{(8)} & 00001100 & Y^{(8)} & 00001101 & Z^{(8)} & 00000001 \\
         X^{(9)} & 00000100 & Y^{(9)} & 00000101 & Z^{(9)} & 00000001 \\
    \end{array} \]
    Measurement of the syndrome $s_P$ will collapse the state to $P \ket{\psi}_L$. For every possible syndrome, there exists a gate $R$ which, when applied, will recover the original state
    \[ RP \ket{\psi}_L = \ket{\psi}_L . \]
\end{topic}

\begin{topic}{quantum-error-mitigation}{quantum error mitigation}
    \emph{Quantum error mitigation} is a method to compensate for readout errors that may occur when measuring \tref{qubit}{qubits}. For every qubit $k = 1, \ldots, n$, define the \textit{assignment matrix} of qubit $k$ as
    \[ A_k = \begin{pmatrix} a_{00} & a_{01} \\ a_{10} & a_{11} \end{pmatrix} \]
    where $a_{ij}$ is the probability that qubit $k$ is measured in state $\ket{j}$ directly after it is initialized in state $\ket{j}$. These values can be estimated experimentally. To mitigate readout errors, suppose a vector of measurements $p \in [0, 1]^{N}$ with $N = 2^n$ is given. That is, $p_i \in [0, 1]$ denotes the probability that the qubits were measured in state $\ket{i}$. Then the error-mitigated vector of measurements is given by
    \[ p' = (A_1^{-1} \otimes A_2^{-1} \otimes \cdots \otimes A_{n}^{-1}) p . \]
    
    More advanced quantum error mitigation can be achieved by also taking into account readout errors that occur due to interactions between different qubits, using multi-qubit assignment matrices.
\end{topic}

\begin{topic}{t2-time}{T2 time}
    The \emph{$T_2^*$ time} (resp. \emph{$T_2^\textup{Hahn}$ time}) of a qubit is the time $T_2$ such that the experiments below result in measuring the qubit in the state $\ket{0}$ with probability
    \[ \PP(\textup{qubit in state }\ket{0}) = A \cdot e^{-t / T_2} + B \]
    for some constants $A$ and $B$.

    \textbf{Experiment for $T_2^*$}:
    \begin{enumerate}[label=(\arabic*)]
        \item Prepare the qubit in state $\ket{0}$.
        \item Apply a \tref{hadamard-gate}{Hadamard gate} to obtain the state $\tfrac{1}{2} (\ket{0} + \ket{1})$.
        \item Wait for some time $t$.
        \item Apply another Hadamard gate.
        \item Measure the qubit.
    \end{enumerate}
    \textbf{Experiment for $T_2^\textup{Hahn}$}:
    \begin{enumerate}[label=(\arabic*)]
        \item Prepare the qubit in state $\ket{0}$.
        \item Apply a \tref{hadamard-gate}{Hadamard gate} to obtain the state $\tfrac{1}{2} (\ket{0} + \ket{1})$.
        \item Wait for some time $t / 2$.
        \item Apply an \tref{pauli-gates}{$X$ gate}.
        \item Wait for some time $t / 2$.
        \item Apply another Hadamard gate.
        \item Measure the qubit.
    \end{enumerate}
\end{topic}

% \begin{example}{t2-time}
%     Consider qubit noise corresponding to a Hamiltonian $H = \delta \ket{1} \bra{1}$ for some $\delta > 0$. The experiment for $T_2^*$ results in the state 
%     \[ \left( \tfrac{1}{2} + \tfrac{1}{2} e^{i \delta t} \right) \ket{0} + \left( \tfrac{1}{2} - \tfrac{1}{2} e^{i \delta t} \right) \ket{1} \]
%     which yields 
%     before measurement, so that $\PP(\ket{0}) = \tfrac{1}{2} + \tfrac{1}{2} \cos(\delta t)$. Hence, $T_2^* = 1 / \delta$ ???
% \end{example}

