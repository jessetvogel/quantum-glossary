\begin{topic}{qubit}{qubit}
    A \emph{qubit} is a quantum system with a two-dimensional Hilbert space $\mathcal{H}$. Its basis vectors are usually denoted by $\ket{0}$ and $\ket{1}$.
    
    An \emph{$n$-qubit quantum system} is a quantum system with Hilbert space $\mathcal{H}^{\otimes n}$. Its basis vectors are usually denoted by $\ket{b_1 b_2 \cdots b_n} = \ket{b_1} \otimes \ket{b_2} \otimes \cdots \otimes \ket{b_n}$.
\end{topic}

\begin{topic}{epr-pair}{Einstein--Podolsky--Rosen (EPR) pair}
    An \emph{Einstein--Podolsky--Rosen (EPR) pair} is a state in the \tref{qubit}{$2$-qubit} quantum system given by
    \[ \frac{1}{\sqrt{2}} (\ket{00} + \ket{11}) . \]
\end{topic}

\begin{example}{epr-pair}
    An EPR pair can be constructed using the following quantum circuit:
    \[ \svg \begin{quantikz}
        & \lstick{$|{0}\rangle$} & \gate{H} & \ctrl{1} & \qw
        \rstick[wires=2]{$\frac{1}{\sqrt{2}} \left( \ket{00} + \ket{11} \right)$} 
        \\
        & \lstick{$|{0}\rangle$} & \qw & \targ{} & \qw
    \end{quantikz} \]
    where $H$ denotes the \tref{hadamard-gate}{Hadamard gate} and $\oplus$ denotes the \tref{cnot-gate}{CNOT gate}.
\end{example}

\begin{topic}{ghz-state}{Greenberger--Horne--Zeilinger (GHZ) state}
    The \emph{Greenberger--Horne--Zeilinger (GHZ) state} is the \tref{qubit}{$3$-qubit} state given by
    \[ \frac{1}{\sqrt{2}} \left( \ket{000} + \ket{111} \right) . \]
    The \emph{generalized GHZ state} is the \tref{qubit}{$n$-qubit} state, for $n \ge 3$ given by
    \[ \frac{1}{\sqrt{2}} \left( \ket{0}^{\otimes n} + \ket{1}^{\otimes n} \right) . \]
\end{topic}

\begin{example}{ghz-state}
    The generalized GHZ state can be realized using a quantum circuit consisting of a \tref{hadamard-gate}{Hadamard gate} and a sequence of \tref{cnot-gate}{CNOT gates}. For instance, the following quantum circuit realizes the GHZ state for $n = 4$:
    \[ \svg \begin{quantikz}
        \lstick{$\ket{0}$} & \gate{H} & \ctrl{1} & \qw & \qw & \qw \\
        \lstick{$\ket{0}$} & \qw & \targ{} & \ctrl{1} & \qw & \qw \\
        \lstick{$\ket{0}$} & \qw & \qw & \targ{} & \ctrl{1} & \qw \\
        \lstick{$\ket{0}$} & \qw & \qw & \qw & \targ{} & \qw
    \end{quantikz} \]
\end{example}

\begin{topic}{bell-states}{Bell's states}
    The \emph{Bell's states} are the four \tref{qubit}{$2$-qubit} states
    \[ \begin{aligned}
        \ket{\Phi^+} &= \frac{1}{\sqrt{2}} (\ket{00} + \ket{11}) \\
        \ket{\Phi^-} &= \frac{1}{\sqrt{2}} (\ket{00} - \ket{11}) \\
        \ket{\Psi^+} &= \frac{1}{\sqrt{2}} (\ket{01} + \ket{10}) \\
        \ket{\Psi^-} &= \frac{1}{\sqrt{2}} (\ket{01} - \ket{10}) .
    \end{aligned} \]
    They form a basis for the $2$-qubit system, called the \textit{Bell basis}.
\end{topic}

\begin{topic}{ancilla-qubit}{ancilla qubit}
    In a quantum circuit, an \emph{ancilla qubit} is an auxiliary \tref{qubit}{qubit}, that is, a qubit which is not used as input or output, but only for intermediate computations.
\end{topic}

% \begin{example}{ancilla-qubit}
%     Ancilla qubits can be used to avoid measurments.
% \end{example}

% \begin{example}{ancilla-qubit}
%     Ancilla qubits are sometimes necessary to keep the circuit reversible.
% \end{example}

\begin{topic}{quantum-signal-processing}{Quantum Signal Processing (QSP)}
    For any $x = \cos \theta \in [-1, 1]$, let $W(x) = \exp(i \theta X) = \begin{pmatrix} x & i \sqrt{1 - x^2} \\ i \sqrt{1 - x^2} & x \end{pmatrix}$ be a unitary gate acting on one \tref{qubit}{qubit}.
    A \emph{Quantum Signal Processing (QSP) sequence} is a quantum circuit of the form
    \[ U(x, \phi_0, \ldots, \phi_n) = \exp(i \phi_0 Z) W(x) \exp(i \phi_1 Z) W(x) \cdots W(x) \exp(i \phi_n Z) \]
    for some $\phi_0, \ldots, \phi_n \in \RR$.
\end{topic}

\begin{example}{quantum-signal-processing}
    \textbf{Theorem} (\href{https://arxiv.org/pdf/2105.02859}{Martyn et al.})
    For any $\phi_0, \ldots, \phi_n \in \RR$, one has
    \[ U(x, \phi_0, \ldots, \phi_n) = \begin{pmatrix}
        P(x) & i Q(x) \sqrt{1 - x^2} \\
        i Q(x)^* \sqrt{1 - x^2} & P(x)^*
    \end{pmatrix} \quad \textup{ for all } x \in [-1, 1] \]
    for some polynomials $P$ and $Q$. Conversely, such $\phi_0, \ldots, \phi_n \in \RR$ exist for any polynomials $P$ and $Q$ satisfying
    \begin{itemize}
        \item $\deg P \le n$ and $\deg Q \le n - 1$,
        \item $P(-x) = (-1)^n P(x)$ and $Q(-x) = (-1)^{n - 1} Q(x)$,
        \item $|P|^2 + (1 - x)^2 |Q|^2 = 1$.
    \end{itemize}
\end{example}

\begin{example}{quantum-signal-processing}
    \begin{itemize}
        \item For $\phi_0 = \phi_1 = \phi_2 = 0$, we have $U(x, 0, 0, 0) = \begin{pmatrix}
            2 x^2 - 1 & 2 i x \sqrt{1 - x^2} \\
            2 i x \sqrt{1 - x^2} & 2 x^2 - 1
        \end{pmatrix}$.
        \item For $\phi_0 = \phi_1 = \phi_2 = \phi_3 = 0$, we have $U(x, 0, 0, 0, 0) = \begin{pmatrix}
            4 x^3 - 3 x & i (4 x^2 - 1) \sqrt{1 - x^2} \\
            i (4 x^2 - 1) \sqrt{1 - x^2} & 4 x^2 - 3 x
        \end{pmatrix}$.
        \item Let $(\phi_0, \phi_1, \phi_2, \phi_3, \phi_4, \phi_5) = (\pi/2, -\eta, 2 \eta, 0, -2 \eta, \eta)$ where $\eta = \tfrac{1}{2} \cos^{-1}(-1/4)$. Then
        \[ U(x, \phi_0, \ldots, \phi_5) = \begin{pmatrix}
            16 x^5 - 20 x^3 + 5 x & i (16 x^4 - 12 x^2 + 1) \sqrt{1 - x^2} \\
            i (16 x^4 - 12 x^2 + 1) \sqrt{1 - x^2} & 16 x^5 - 20 x^3 + 5 x
        \end{pmatrix} \]
        up to a global phase factor. This sequence is known as the \textit{BB1 pulse sequence}.
    \end{itemize}
\end{example}
