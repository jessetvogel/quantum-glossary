\begin{topic}{deutsch-jozsa-algorithm}{Deutsch--Jozsa algorithm}
    The \emph{Deutsch--Jozsa algorithm} is a quantum algorithm to solve the following problem:
    
    \textbf{Problem}: Given $x \in \{ 0, 1 \}^N$ for some $N = 2^n$, such that either all $x_i$ are equal (in this case, $x$ is called `constant'), or $N / 2$ of the $x_i$ are $0$ and $N / 2$ are $1$ (in this case, $x$ is called `balanced'). Determine whether $x$ is constant or balanced.
    
    \textbf{Algorithm}: Denote by $\mathcal{O}$ the query which acts as $\mathcal{O} \ket{i} = (-1)^{x_i} \ket{i}$ for all $i \in \{ 0, 1 \}^n$.
    \begin{enumerate}[label=(\arabic*)]
        \item Start with the $n$-qubit state $\ket{0^n}$.
        \item Apply a \tref{hadamard-gate}{Hadamard gate} to each qubit to obtain the state
        \[ \frac{1}{\sqrt{2^n}} \sum_{i \in \{ 0, 1 \}^n } \ket{i} . \]
        \item Apply the query $\mathcal{O}$ to obtain the state
        \[ \frac{1}{\sqrt{2^n}} \sum_{i \in \{ 0, 1 \}^n} (-1)^{x_i} \ket{i} . \]
        \item Apply a Hadamard gate to each qubit to obtain the state
        \[ \frac{1}{2^n} \sum_{i \in \{ 0, 1 \}^n} (-1)^{x_i} \sum_{j \in \{ 0, 1 \}^n} (-1)^{i \cdot j} \ket{j} . \]
        In particular, the amplitude of the $\ket{0^n}$-state in the final superposition is
        \[ \frac{1}{2^n} \sum_{i \in \{ 0, 1 \}^n} (-1)^{x_i} = \begin{cases}
            1 & \textup{if } x_i = 0 \textup{ for all } i , \\
            -1 & \textup{if } x_i = 1 \textup{ for all } i , \\
            0 & \textup{if } x \textup{ is balanced} .
        \end{cases} \]
        Measuring will yield $\ket{0^n}$ if $x$ is constant, and any other state if $x$ is balanced.
    \end{enumerate}
\end{topic}

\begin{topic}{bernstein-vazirani-algorithm}{Bernstein--Vazirani algorithm}
    The \emph{Bernstein--Vazirani algorithm} is a quantum algorithm to solve the following problem:
    
    \textbf{Problem}: Given $x \in \{ 0, 1 \}^N$ for some $N = 2^n$, such that $x_i = (i \cdot s) \mod 2$ for some unknown $s \in \{ 0, 1 \}^n$. The problem is to determine $s$.

    \textbf{Algorithm}: Denote by $\mathcal{O}$ the query which acts as $\mathcal{O} \ket{i} = (-1)^{x_i} \ket{i}$ for all $i \in \{ 0, 1 \}^n$.
    \begin{enumerate}[label=(\arabic*)]
        \item Start with the $n$-qubit state $\ket{0^n}$.
        \item Apply a \tref{hadamard-gate}{Hadamard gate} to each qubit to obtain the state
        \[ \frac{1}{\sqrt{2^n}} \sum_{i \in \{ 0, 1 \}^n } \ket{i} . \]
        \item Apply the query $\mathcal{O}$ to obtain the state
        \[ \frac{1}{\sqrt{2^n}} \sum_{i \in \{ 0, 1 \}^n} (-1)^{x_i} \ket{i} = \frac{1}{\sqrt{2^n}} \sum_{i \in \{ 0, 1 \}^n} (-1)^{i \cdot s} \ket{i} . \]
        \item Apply a Hadamard gate to each qubit to obtain the state $\ket{s}$.
        \item Measure the qubits to find $s$.
    \end{enumerate}
\end{topic}

\begin{topic}{simon-algorithm}{Simon's algorithm}
    \emph{Simon's algorithm} is a quantum algorithm to solve the following problem:

    \textbf{Problem}: Given $x = (x_0, \ldots, x_{N - 1})$ for some $N = 2^n$, where $x_i \in \{ 0, 1 \}^n$, with the property that there is some unknown non-zero $s \in \{ 0, 1 \}^n$ such that $x_i = x_j$ if and only if ($i = j$ or $i = j \oplus s$). Determine $s$.

    Note that $x$, viewed as a function from $\{ 0, \ldots, N - 1 \}$ to $\{ 0, \ldots, N - 1 \}$ is a $2$-to-$1$ function, whose $2$-to-$1$-ness is encoded by the unknown \textit{mask} $s$.

    \textbf{Algorithm}: Denote by $\mathcal{O}$ the query which acts as $\mathcal{O} \ket{i} \ket{y} = \ket{i} \ket{y \oplus x_i}$ for all $i \in \{ 0, 1 \}^n$.
    \begin{enumerate}[label=(\arabic*)]
        \item Start with the $2n$-qubit state $\ket{0^n} \otimes \ket{0^n}$.
        \item Apply \tref{hadamard-gate}{Hadamard gates} to the first $n$ qubits to obtain the state
        \[ \frac{1}{\sqrt{2^n}} \sum_{i \in \{ 0, 1 \}^n} \ket{i} \otimes \ket{0^n} . \]
        \item Apply the query $\mathcal{O}$ to obtain the state
        \[ \frac{1}{\sqrt{2^n}} \sum_{i \in \{ 0, 1 \}^n} \ket{i} \otimes \ket{x_i} . \]
        \item Measure the last $n$ qubits to obtain some value $x_i$. The state will now have collapsed to
        \[ \frac{1}{\sqrt{2}} (\ket{i} + \ket{i \oplus s}) \otimes \ket{x_i} . \]
        \item Apply Hadamard gates to the first $n$ qubits to obtain the state
        \[ \begin{aligned}
            \frac{1}{\sqrt{2^{n + 1}}} \left( \sum_{j \in \{ 0, 1 \}^n} (-1)^{i \cdot j} \ket{j} + \sum_{j \in \{ 0, 1 \}^n} (-1)^{(i \oplus s) \cdot j} \ket{j} \right) \\
            = \frac{1}{\sqrt{2^{n + 1}}} \left( \sum_{j \in \{ 0, 1 \}^n} (-1)^{i \cdot j} (1 + (-1)^{s \cdot j}) \ket{j} \right) .
        \end{aligned} \]
        \item Measure the first $n$ qubits to obtain some value $j \in \{ 0, 1 \}^n$. Since $\ket{j}$ has non-zero amplitude if and only if $s \cdot j \equiv 0 \mod 2$, this measurement yields a random element uniformly distributed from the set $\{ j \in \{ 0, 1 \}^n \mid s \cdot j \equiv 0 \mod 2 \}$. Such an element corresponds to a linear equation that gives information about $s$.
        \item Repeat the above until $n - 1$ linearly independent equations are obtained.
        \item Solve for $s$.
    \end{enumerate}
    Simon's algorithm uses $O(n)$ calls to $\mathcal{O}$, and polynomially in $n$ many operations to solve for $s$.
\end{topic}

\begin{topic}{kitaev-phase-estimation-algorithm}{Kitaev's phase estimation algorithm}
    Given a unitary $U$ and eigenvector $\ket{\psi}$ whose eigenvalue is $\lambda = \exp(2 \pi i \phi)$ for some $\phi \in [0, 1)$, \emph{Kitaev's phase estimation algorithm} is a quantum algorithm to compute $\phi$, given $U$ and $\ket{\psi}$.

    \textbf{Algorithm}: For simplicity, assume that $\phi$ can be expressed with $n$ bits of precision, that is, $\phi = \phi_1 / 2 + \phi_2 / 4 + \ldots + \phi_i / 2^n$.
    \begin{enumerate}[label=(\arabic*)]
        \item Start with the $2n$-qubit state $\ket{0^n} \otimes \ket{\psi}$.
        \item For $N = 2^n$, apply the \tref{quantum-fourier-transform}{quantum Fourier transform} $F_N$ to the first $n$ qubits to obtain the state
        \[ \frac{1}{\sqrt{N}} \sum_{j = 0}^{N - 1} \ket{j} \otimes \ket{\psi} . \]
        (Alternatively, applying the $n$-fold \tref{hadamard-gate}{Hadamard gate} $H^{\otimes n}$ has the same effect.)
        \item Apply the operation $\ket{j} \otimes \ket{\psi} \mapsto \ket{j} \otimes U^j \ket{\psi} = \exp(2 \pi i \phi j) \ket{j} \otimes \ket{\psi}$. This can be implemented, for instance, using controlled $U$ gates. This yields the state
        \[ \frac{1}{\sqrt{N}} \sum_{j = 0}^{N - 1} \exp(2 \pi i \phi j) \ket{j} \otimes \ket{\psi} . \]
        \item Apply the inverse quantum Fourier transform $F_N^{-1}$ to the first $n$ qubits to obtain the state
        \[ \ket{\phi_1 \phi_2 \cdots \phi_n} \otimes \ket{\psi} . \]
    \end{enumerate}
    The value of $\phi$ can now be obtained by measuring the first $n$ qubits.
\end{topic}

% \begin{topic}{shor-algorithm}{Shor's algorithm}
    
% \end{topic}

\begin{topic}{grover-search-algorithm}{Grover's search algorithm}
    \emph{Grover's search algorithm} is a quantum algorithm to solve the following problem:

    \textbf{Problem}: Given $x \in \{ 0, 1 \}^N$ for some $N = 2^n$, find some $i \in \{ 0, 1 \}^n$ such that $x_i = 1$.

    \textbf{Algorithm}: Denote by $\mathcal{O}$ the query which acts as $\mathcal{O} \ket{i} = (-1)^{x_i} \ket{i}$ for all $i \in \{ 0, 1 \}^n$. Denote by $R$ the unitary operation given by $R \ket{0^n} = \ket{0^n}$ and $R \ket{i} = - \ket{i}$ for $i \ne 0^n$. Define the \textit{Grover iterate} to be the operation $\mathcal{G} = H^{\otimes n} R H^{\otimes n} \mathcal{O}$, where $H$ is the \tref{hadamard-gate}{Hadamard gate}.
    \begin{enumerate}[label=(\arabic*)]
        \item Start with the $n$-qubit state $\ket{0^n}$.
        \item Apply a Hadamard gate to each qubit to obtain the uniform state
        \[ \ket{U} = \frac{1}{\sqrt{2^n}} \sum_{i \in \{ 0, 1 \}^n } \ket{i} . \]
        \item Write $\ket{U} = \sin(\theta_0) \ket{G} + \cos(\theta_0) \ket{B}$ for $\theta_0 = \arcsin(\sqrt{t / N})$ and
        \[ \ket{G} = \frac{1}{\sqrt{t}} \sum_{i \mid x_i = 1} \ket{i}, \quad \ket{B} = \frac{1}{\sqrt{N - t}} \sum_{i \mid x_i = 0} \ket{i} , \]
        where $t$ is the number of $x_i$ equal to $1$.
        \item Observe that the Grover iterate $\mathcal{G}$ is the composite of $\mathcal{O}$ and $H^{\otimes n} R H^{\otimes n} = 2 \ket{U} \bra{U} - I$. In the $2$-dimensional plane spanned by $\ket{G}$ and $\ket{B}$, this corresponds to a reflection through $\ket{B}$ followed by a reflection through $\ket{U}$. Therefore,
        \[ \mathcal{G}^k \ket{U} = \sin(\theta_k) \ket{G} + \cos(\theta_k) \ket{B} \quad \textup{ with } \theta_k = (2k + 1) \theta_0 \]
        for any $k \ge 0$.
        \item Choose $k$ such that applying $k$ Grover iterates maximizes the probability to measure some $i$ with $x_i = 1$. Explicitly, this probability is $P_k = \sin((2k + 1) \theta_0)^2$, so choose $k$ to be the integer closest to $\tilde{k} = \tfrac{\pi}{4} \theta_0 - \tfrac{1}{2}$.
        \item Apply $k$ Grover iterates and measure the qubits to obtain some $i$. The failure probability is
        \[ \begin{aligned}
            1 - P_k
                &= \cos((2k + 1) \theta_0)^2 = \cos((2 \tilde{k} + 1) \theta_0 + 2 (k - \tilde{k}) \theta_0)^2 \\
                &= \cos(\pi/2 + 2 (k - \tilde{k}) \theta_0)^2 = \sin(2(k - \tilde{k}) \theta_0)^2 \le \sin(\theta_0)^2 = \frac{t}{N} .
        \end{aligned} \]
    \end{enumerate}
    Since $\arcsin(\theta_0) \ge \theta_0$, the number of Grover iterates is $k \le \tfrac{\pi}{4} \theta_0 \le \tfrac{\pi}{4} \sqrt{N/t}$. When $t$ is unknown, the above algorithm is run for different guesses of $k$. The expected number of queries to use is $O(\sqrt{N / t})$.
\end{topic}

\begin{topic}{hhl-algorithm}{Harrow--Hassidim--Lloyd (HHL) algorithm}
    The \emph{Harrow--Hassidim--Lloyd (HHL) algorithm} is a quantum algorithm to solve the following problem:

    \textbf{Problem}: Given an $N \times N$ Hermitian matrix $A$ and an $N$-dimensional unit vector $b$, for some $N = 2^n$, find an $n$-qubit state $\ket{x} = \frac{1}{\norm{x}} \sum_{i = 0}^{N - 1} x_i \ket{i}$ such that $Ax = b$.

    \textbf{Algorithm}: As $A$ is Hermitian, it can be decomposed as $A = \sum_{j = 0}^{N - 1} \lambda_j v_j v_j^T$ for orthonormal eigenvectors $v_j$ with corresponding eigenvalues $\lambda_j$. Assume $A$ is \textit{well-conditioned}: the ratio $|\lambda_\textup{max} / \lambda_\textup{min}|$ between the largest and smallest eigenvalue of $A$ (in absolute value) is at most some $\kappa$ (the smaller $\kappa$ is, the better the algorithm performs). Moreover, assume $|\lambda_\textup{min}| \ge 1 / \kappa$ and $|\lambda_\textup{max}| \le 1$. Decompose $b$ with respect to the eigenvectors $v_j$ as $b = \sum_{j = 0}^{N - 1} \beta_j v_j$.
    \begin{enumerate}[label=(\arabic*)]
        \item Choose $\ell \ge 1$ large enough for sufficient precision.
        \item Prepare the $(n + \ell + 1)$-qubit state $\ket{b} \otimes \ket{0^\ell} \otimes \ket{0}$.
        \item Apply \tref{kitaev-phase-estimation-algorithm}{Kitaev's phase estimation algorithm}, using Hamiltonian simulation to implement the unitary $U = \exp(i A)$ and powers of $U$, on the first $n + \ell$ qubits to estimate the eigenvalues $\lambda_j$, obtaining the state
        \[ \sum_{j = 0}^{N - 1} \beta_j \ket{v_j} \otimes \ket{\tilde{\lambda}_j} \otimes \ket{0} , \]
        where $\tilde{\lambda}_j$ is an $\ell$-bit approximation of $\lambda_j$.
        \item Rotate the last qubit conditioned on the middle $\ell$ qubits (that is, conditioned on $\lambda_j$) to obtain
        \[ \sum_{j = 0}^{N - 1} \beta_j \ket{v_j} \otimes \ket{\tilde{\lambda}_j} \otimes \left( \tfrac{1}{\kappa \lambda_i} \ket{0} + \sqrt{1 - \tfrac{1}{(\kappa \lambda_i)^2}} \ket{1} \right) . \]
        \item Apply the phase estimation algorithm in reverse to revert the middle $\ell$ qubits to the state $\ket{0^\ell}$. Now, disregard the middle $\ell$ qubits, obtaining the state
        \[ \sum_{j = 0}^{N - 1} \beta_j \ket{v_j} \otimes \left( \tfrac{1}{\kappa \lambda_i} \ket{0} + \sqrt{1 - \tfrac{1}{(\kappa \lambda_i)^2}} \ket{1} \right) = \frac{1}{\kappa} \underbrace{\sum_{j = 0}^{N - 1} \beta_j \lambda_j^{-1} \ket{v_j}}_{\propto \ket{x}} \otimes \ket{0} + \ket{\phi} \otimes \ket{1} \]
        for some (unnormalized) state $\ket{\phi}$.
        \item Since $\sum_{j = 0}^{N - 1} |\beta_j / \lambda_j|^2 \ge \sum_{j = 0}^{N - 1} |\beta_j|^2 = 1$, the norm of the part of the state ending in qubit $\ket{0}$ is at least $1 / \kappa$. Hence, applying $O(\kappa)$ rounds of amplitude amplification to have amplitude essentially one.
        \item Measure (with high probability) the last qubit to be $0$. The first $n$ qubits will be in a state close to $\ket{x}$.
    \end{enumerate}
\end{topic}

% \begin{example}{hhl-algorithm}
%     Actually $A$ need not be Hermitian:
%     Without loss of generality, assume $A$ is Hermitian: otherwise, replace $A$ by $2N \times 2N$ Hermitian matrix $\left( \begin{smallmatrix} 0 & A \\ A^\dag & 0 \end{smallmatrix} \right)$, and $b$ by $\left( \begin{smallmatrix} b \\ 0 \end{smallmatrix} \right)$. (TODO: correct?)
% \end{example}
